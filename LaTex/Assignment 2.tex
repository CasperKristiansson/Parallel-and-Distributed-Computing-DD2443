\documentclass{article}
\usepackage{tabularx}
\usepackage{graphicx}
\usepackage{dirtytalk}
\usepackage{pgfplotstable} 
\usepackage{pgfplots}
\usepackage{datatool}
\usepackage{siunitx}
\usepackage[hyphens]{url}
\usepackage{hyperref}
\usepackage{graphicx}
\usepackage{microtype}
\usepackage{float}
\usepackage[style=ieee]{biblatex}
\usepackage{listings}
\usepackage{xcolor}
\usepackage[normalem]{ulem}
\useunder{\uline}{\ul}{}

\addbibresource{main.bib}

\hypersetup{
    colorlinks=true,
    linkcolor=blue,
    filecolor=blue,      
    urlcolor=blue,
    citecolor=blue,
}

\pgfplotsset{compat=1.18}

\title{\textbf{Parallel and Distributed Computing\\DD2443 - Pardis24\\Exercises for Lecture 2}}
\author{Name: Casper Kristiansson}
\date{\today}

\begin{document}

\setlength\parindent{0pt}
\setlength{\parskip}{\bigskipamount}

\maketitle

\section*{Exercise 1}
\subsection*{Question}
\textbf{Consider the class \texttt{VolatileCounters} in Figure 1. The ``volatile'' declaration ensures that variable reads and writes occur in a one-at-a-time sequential order (as one might expect) when accessed by each of the parallel threads. Assume \texttt{actor1} and \texttt{actor2} are invoked to run in parallel.}

\begin{enumerate}
    \item \textbf{What are the relevant events?}
    \item \textbf{Sketch the state machine graph (only parts, it is quite large). Include all events you have identified in 1.}
    \item \textbf{Sketch a few traces of the state machine.}
    \item \textbf{List the relevant intervals for the program.}
    \item \textbf{Which are the possible final values for \texttt{x}? Explain your reasoning very carefully.}
\end{enumerate}

\textbf{Note that this is more subtle than it looks. Are you sure that you have the right solution?}

\subsection*{Answer}



\section*{Exercise 2}
\subsection*{Question}
\textit{\textbf{HSLS Exercise 2.3 (Flaky Computer Corporation) Use the method presented in class and in the textbook to solve this.}}

\textbf{Programmers at the Flaky Computer Corporation designed the protocol shown in Fig. 2.16 to achieve \textit{n}-thread mutual exclusion. For each question, either sketch a proof or display an execution where it fails.}

\begin{itemize}
    \item \textbf{Does this protocol satisfy mutual exclusion?}
    \item \textbf{Is this protocol starvation-free?}
    \item \textbf{Is this protocol deadlock-free?}
    \item \textbf{Is this protocol livelock-free?}
\end{itemize}

\subsection*{Answer}


\section*{Exercise 3}
\subsection*{Question}
\textbf{\textit{HSLS Exercise 2.8 (Fast path lock) Use the method presented in class and in the textbook to solve this.}}

\textbf{In practice, almost all lock acquisitions are uncontended, so the most practical measure of a lock’s performance is the number of steps needed for a thread to acquire a lock when no other thread is concurrently trying to acquire the lock. Scientists at Cantaloupe-Melon University have devised the following “wrapper” for an arbitrary lock, shown in Fig. 2.18. They claim that if the base \texttt{Lock} class provides mutual exclusion and is starvation-free, so does the \texttt{FastPath} lock, but it can be acquired in a constant number of steps in the absence of contention. Sketch an argument why they are right, or give a counterexample.}

\subsection*{Answer}

\newpage
\printbibliography

\end{document}